\documentclass[12pt,a4paper]{article}
\usepackage[utf8]{inputenc}
\usepackage[brazil]{babel}

\title{Resumo de Medidas}
\date{Março, 18,2019}
\author{Germano Barcelos}

\begin{document}
    \maketitle
    \begin{table}[h]
        \centering
        \caption{Unidade de Três Grandezas Fundamentais do SI}
        \vspace{0.5cm}
        \begin{tabular}{lll}
            \hline
            Grandeza & Nome da Unidade & Símbolo da Unidade \\
            \hline
            Comprimento & metro & m \\
            Tempo & segundo & s\\
            Massa & quilograma & kg \\
            \hline
        \end{tabular}
    \end{table}

    \begin{table}[h]
        \centering
        \caption{Prefixos das Unidades do SI}
        \vspace{0.5cm}
        \begin{tabular}{lll}
            \hline
            Fator & Prefixo & Símbolo \\
            \hline
            $10^9$ & giga- & G \\
            $10^6$ & mega- & M \\
            $10^3$ & quilo- & Q \\
            $10^{-2}$ & centi- & c \\
            $10^{-3}$ & mili- & m \\
            $10^{-6}$ & micro- & $\mu$ \\
            $10^{-9}$ & nano- & n \\
            $10^{-12}$ & pico- & p
        \end{tabular}
    \end{table}  
    
    \section*{A Medição na Física}
    A física se baseia na medição de grandezas físicas. Algumas grandezas físicas, como comprimento, tempo e massa, foram escolhidas como grandezas fundamentais; cada uma foi definida através de um padrão e recebeu uma unidade de medida (como metro, segundo e quilograma). Outras grandezas físicas são definidas em termos das grandezas fundamentais e de seus padrões e unidades.
    \section*{Mudança de Unidades}
    A conversão de unidades pode ser feita usando o meodo de conversão em cadeia, no qual os dados originais são multiplicados sucessivamente por fatores de conversão unitários e as unidades são manipuladas como quantidades algébricas até que apenas as unidades desejadas permaneçam.
    \section*{Massa Específica}
    A massa específica $\rho$ de uma substância é a massa por unidade de volume:
    \begin{equation}
        \rho = m/V
    \end{equation}
    \section*{Unidade de Massa Atômica}
    \begin{equation}
        u=1.66053886 * 10^{-27}kg
    \end{equation}
\end{document}